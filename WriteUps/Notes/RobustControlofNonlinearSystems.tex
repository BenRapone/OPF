\documentclass[11pt]{article}
\usepackage{amsmath,amstext,amsfonts,amssymb,amsthm,epsfig,epstopdf,url,array}
\usepackage[margin=1in]{geometry}
\usepackage{xcolor}
\usepackage{graphicx}
\usepackage{times}

\theoremstyle{plain}
\newtheorem{thm}{Theorem}[section]
\newtheorem{lem}[thm]{Lemma}
\newtheorem{prop}[thm]{Proposition}
\newtheorem{cor}[thm]{Corollary}

\theoremstyle{definition}
\newtheorem{con}{Conjecture}[section]
\newtheorem{exa}{Example}[section]
\newtheorem*{sol}{Solution}
\newtheorem{cdef}{Definition}[section]


\theoremstyle{remark}
\newtheorem*{rem}{\color{red}\textbf{Remark}}
\newtheorem*{note}{\color{blue}\textbf{Note}}
\usepackage{qtree}

\begin{document}
\begin{center}
\Large Robust optimization and control for \\ infrastructure network applications
\end{center}

\section*{Abstract}

Optimal power flow (OPF) is the central optimization problem in electric power grids. As infrastructure networks like the power grid modernize and undergo transformation to enable the integration of sustainable energy technologies, the sources of uncertainty in these networks begin to grow. As uncertainty increases, the set of scenarios that needs to be considered in the offline studies grows rapidly, and even with a large number of scenarios, it is not assured that the settings computed offline guarantee safety (non-violation of system constraints on voltages/flows etc.) under real system operations. This makes the problems of optimal operation and resource utilization challenging problems. In this article we present novel algorithms that produce robust control policies with feasibility and performance bound guarantees.

\section{Introduction}

\section{Canonical Model of Infrastructure Networks}







\end{document}

