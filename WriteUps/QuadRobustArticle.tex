\documentclass[11pt]{article}
\usepackage{amsmath,amstext,amsfonts,amssymb,amsthm,epsfig,epstopdf,url,array}
\usepackage[margin=1in]{geometry}
\usepackage{xcolor}
\usepackage{graphicx}
\usepackage{times}

\theoremstyle{plain}
\newtheorem{thm}{Theorem}[section]
\newtheorem{lem}[thm]{Lemma}
\newtheorem{prop}[thm]{Proposition}
\newtheorem{cor}[thm]{Corollary}
\newtheorem{defn}[thm]{Definition}
\newtheorem{claim}[thm]{Claim}

\theoremstyle{definition}
\newtheorem{con}{Conjecture}[section]
\newtheorem{exa}{Example}[section]
\newtheorem*{sol}{Solution}
\newtheorem{cdef}{Definition}[section]


\theoremstyle{remark}
\newtheorem{rem}{\textbf{Remark}}
\newtheorem*{note}{\color{blue}\textbf{Note}}
\usepackage{qtree}


\usepackage{hyperref}

\usepackage[nameinlink,noabbrev,capitalize]{cleveref} 
\crefalias{subequation}{equation}
\crefalias{thm}{theorem}


% to make cleveref print ``Lemma'' for lemma
\let\oldlemma\lem
\renewcommand{\lem}{%
  \crefalias{thm}{lem}% Theorem counter now looks like Lemma
  \oldlemma}
\Crefname{lem}{Lemma}{Lemmas}

% to make cleveref print ``Definition for definition
\let\olddefn\defn
\renewcommand{\defn}{%
  \crefalias{thm}{defn}% Theorem counter now looks like Definition
  \olddefn}
\Crefname{defn}{Definition}{Definitions}

% to make cleveref print ``Remark for remark
\let\oldrem\rem
\renewcommand{\rem}{%
  \crefalias{thm}{rem}% Theorem counter now looks like Remark
  \oldrem}
\Crefname{rem}{Remark}{Remarks}

% to make cleveref print ``Corollary for corollary
\let\oldcor\cor
\renewcommand{\cor}{%
  \crefalias{thm}{cor}% Theorem counter now looks like Corollary
  \oldcor}
\Crefname{cor}{Corollary}{Corollaries}

% to make cleveref print ``Claim for claim
\let\oldclaim\claim
\renewcommand{\claim}{%
  \crefalias{thm}{claim}% Theorem counter now looks like Claim
  \oldclaim}
\Crefname{claim}{Claim}{Claims}

% to make cleveref print ``Proposition for prop
\let\oldprop\prop
\renewcommand{\prop}{%
  \crefalias{thm}{prop}% Theorem counter now looks like Prop
  \oldprop}
\Crefname{prop}{Proposition}{Propositions}

% to make cleveref print ``Conjecture for conj
\let\oldcon\con
\renewcommand{\con}{%
  \crefalias{thm}{con}% Theorem counter now looks like Con
  \oldcon}
\Crefname{con}{Conjecture}{Conjectures}

\bibliographystyle{plain}

\begin{document}
\begin{center}
\textbf{Minimum Homotopy Area}
\end{center}



\section*{Abstract}
Systems of quadratic equations appear in several applications from analysis of stochastic processes, machine learning to physical infrastructure systems like the power grid. In recent years, several algorithms have been proposed for the solution of quadratic systems, along with analyses of conditions under which they work. However, in many applications, parameters appearing in the quadratic system of equations are not known perfectly. In these cases, it is of interest to study robust solvability of quadratic systems, that is, whether the quadratic system of equations has a solution (within a certain set) for all realizations within a certain error bound of a given nominal value of the parameters. This problem is NP-hard in general, but we identify special cases that are tractable. Furthermore, we develop a general technique to produce inner and outer bounds on the maximum error bound for which one can guarantee robust solvability (the radius of robust solvability).
The techniques we use combine ideas from constraint programming, convex optimization and topological degree theory. We evaluate our approach numerically on several quadratic systems constructed from the AC power flow equations that describe the steady state of the power grid. The results show that our approach can produce tight lower and upper bounds on the radius of robust solvability.


\section{Introduction}
This paper studies quadratic systems of equations with parameters. More concretely, we study a system of $n$ quadratic equations $F(x)=u$ where $F: \mathbb{R}^n \mapsto \mathbb{R}^n, x,u \in \mathbb{R}^n$ and $F$ is quadratic in $x$. We are interested in situations where the parameters $u$ are uncertain and we are still interested in guaranteeing that there is a solution to $F(x) = u$ for $x$ within limits on $x$. Questions of this type arise in a variety of applications from analysis of stochastic processes to infrastructure networks like the power grid and the as grid. For example, in binary markov trees, the parameter $u$ represents the initial probabilities of the Markov chain.

\section{Problem formulation}

\begin{itemize}
\item[] $\mathbb{R}$: Set of real numbers, $\mathbb{R}^n$: $n$-dimensional Euclidean space 
\item[] $\mathbb{S}^n$: Set of $n \times n$ symmetric matrices 
\item[] $\mathbb{R}^{n \times m}$: Set of $n \times m$ real matrices
\item[] $x \in \mathbb{R}^n$): $x$ is a real vector variable
\item[] $M \geq 0$ (for $M \in \mathbb{R}^{n \times n}$): $M$ is a matrix with all entries non-negative  
\item[] $A \in \mathbb{R}^{n\times n}$: $A$ is a potentially sparse condition matrix
\item[] $b \in \mathbb{R}^n$: $b$ is a non-zero vector with all components non-negative
\end{itemize}

We study systems of quadratic equations of the form
\begin{align}
& Q(x)+Lx=u\label{eq:Quad}
\end{align}
where $Q: \mathbb{R}^n \mapsto \mathbb{R}^n$ is a vector-valued quadratic function, that is, there exist symmetric matrices $\mathcal{M}^1,\ldots,\mathcal{M}^{n} \in \mathbb{S}^n$ such that
\[[Q(x)]_i = x^t \mathcal{M}_{i} x \quad \forall i \in [n]\]
and $L \in \mathbb{R}^{n\times n}$ is an $n \times n$ matrix and $u \in \mathbb{R}^n$ is a vector. We are interested in solutions to this system of equations such that
\begin{align}
(Ax)_i\leq b_i \quad \forall i \in [n]\label{eq:xLimits}
\end{align}
However, the parameter $u$ is uncertain and only known upto certain error bounds
\begin{align}
u^{\min}_i=u_i^\star-e_i \leq u_i \leq =u_i^\star+e_i=u^{\max}_i \quad \forall i \in [n] \label{eq:uLimits}
\end{align}
where $u^\star$ is a forecast for $u$ and $e$ denotes the error bounds associated with the forecast. For example, in the case of quadratic equations appearing in infrastructure networks like the power grid, $u_i$ represents uncertain power generation or consumption (for example uncertain weather-dependent power sources like solar or wind power). In the case of stochastic processes, $u^\star$ represents an initial state distribution.

\begin{cdef}[Robust solvability problem]
Determine whether for all values of $u$ satisfying \eqref{eq:uLimits}, the system of equations \eqref{eq:Quad} has a solution for $x$ satisfying the constraints \eqref{eq:xLimits}. If this is true, the system \eqref{eq:Quad},\eqref{eq:xLimits},\eqref{eq:uLimits} is said to be \emph{Robust feasible}.
\end{cdef}

\section{Main technical results}
We now describe the main technical results of this paper. In the first subsection we describe the setting under which the problem can be solved using the results which follow. Our results take advantage of the well studied area of topological degree theory. For an introduction to topological degree theory see {\color{blue}{NEED REFS}}. Suffice it then to say that should $\Omega\in\mathbb{R}^{n}$ be open and bounded, $F:\Omega\rightarrow \mathbb{R}$ continuous, and $F(x)\neq y \quad \forall x\in\partial\Omega$ for some $y\in\mathbb{R}^n$, then $d\left(\Omega,F,y\right)\in\mathbb{Z}$ is defined. 
As in {\color{blue}{Frommer, Hoxha and Lang Thm1}} we utilize the following two results which we form into the following theorem.
\begin{thm} \ \\
\label{thm:Deg}
\begin{itemize}
\item[(i)] If $H : [0,1]\times\bar{\Omega}\rightarrow\mathbb{R}^n$ is continuous such that $H(t,x)\neq y \quad \forall t\in[0,1], \quad x\in\partial\Omega$ then $d\left(\Omega,H(t,\cdot),y\right)$ does not depend on $t$.
\item[(ii)] If $d(\Omega,F,y)\neq 0$, then there exists $x\in\Omega$ s.t. $F(x)=y$.
\end{itemize}

\end{thm}

Our problem necessitates $\Omega=\{x| Ax< b\}$, $F(x)=Q(x)+Lx$ and $y=u$. We define a homotopy $H : [0,1]\times\bar{\Omega}\rightarrow\mathbb{R}^n$ as 
\begin{align}
H(t,x) = F(x)-t\cdot u \label{eq:Homo}
\end{align}
Observe that by construction $F(0)=0$ and thus $d(\Omega,F(x),0)\neq 0$. If we wish to guarantee a solution to $F(x)=u$ for some $u\in[u_{min},u_{max}]$ then this is equivalent by (ii) of \cref{thm:Deg} to showing $d(\Omega,H(1,x),0)\neq 0$. However, by (i) of \cref{thm:Deg}, if it can be shown that $H(t,x)\neq 0, \  \forall t\in[0,1], \  x\in\partial\Omega$, then $d\left(\Omega,H(t,x),0\right)$ does not depend on $t$ and it follows that $d\left(\Omega,H(t,x),0\right)=d\left(\Omega,H(0,x),0\right)=d(\Omega,F(x),0)\neq 0$. Therefore by \cref{thm:Deg}, it follows that the robust solvability problem can be determined by validating/invalidating the following statement.
\begin{align}
\not\exists x\in\partial\Omega, \ s.t. \ F(x)-u=0 \ \ for \ any \ \ u\in[u_{min},u_{max}]. \label{eq:RSForm}
\end{align}

We now provide results which help derive methods for determining inner and outer approximations on the maximum error bound for which one can guarantee robust solvability, i.e. we provide methodology for finding inner and outer approximations on $u_{min}$ and $u_{max}$ s.t. \eqref{eq:RSForm} is validated.\\

\begin{lem} \ \\
\label{MainInEqLem}
Let $X\subset\mathbb{R}^n$ be closed and $\bold{f}:\mathbb{R}^n\rightarrow\mathbb{R}^n$ be continuous. If $\bold{f}$ has no singularities and $$\min\limits_{||\bold{\lambda}||=1}\max\limits_{x\in X}\ \bold{\lambda}^T\bold{f}(x)$$
obtains its optimal at $\bold{\hat{x}}$,$\bold{\lambda}_{\bold{\hat{x}}}$ then $\bold{f}(\bold{\hat{x}})\in \partial \bold{f}(X)$. 

\begin{proof} \ \\
(By contradiction) Assume $\bold{f}(\bold{\hat{x}})\in \bold{f}(X)\setminus\partial \bold{f}(X)$. Let $\theta$ the angle between $\bold{\lambda}_{\bold{\hat{x}}}$ and $\bold{\hat{x}}$. Thus $\min\limits_{||\bold{\lambda}||=1}\max\limits_{x\in X}\ \bold{\lambda}^T\bold{f}(x)=\bold{\lambda}_{\bold{\hat{x}}}^T|\bold{f}(\bold{\hat{x}})|=|\bold{f}(\bold{\hat{x}})|\cos(\theta)$. Since $X\in\mathbb{R}^n$ is closed it is thus compact which implies $\bold{f}(X)$ is also compact. Thus $\exists r>0$ s.t. (ball of radius $r$ centered at $\bold{f}(\bold{\hat{x}})$) $B_r(\bold{f}(\bold{\hat{x}}))\in \bold{f}(X)\setminus\partial \bold{f}(X)$. Let $y$ be the antipodal point on $\partial B_r(\bold{f}(\bold{\hat{x}}))$ to the point of intersection between the line segment connecting the origin to $\bold{f}(\bold{\hat{x}})$ and $B_r(\bold{f}(\bold{\hat{x}}))$. It follows then that $|y|>|\bold{f}(\bold{\hat{x}})|$ and $\theta$ is the angle between $\bold{\lambda}_x$ and $y$.   Let $\bold{x^*}\in X$ s.t. $\bold{f}(\bold{x^*})=y$, such a $\bold{x^*}$ exists as $\bold{f}(X)$ is compact. Therefore $\bold{\lambda}_{\bold{\hat{x}}}^T|\bold{f}(\bold{x^*})|=|\bold{f}(\bold{x^*})|\cos(\theta)>|\bold{f}(\bold{\hat{x}})|\cos(\theta)=\bold{\lambda}_{\bold{\hat{x}}}^T|\bold{f}(\bold{\hat{x}})|$ which is a contradiction. 
The lemma now follows.
\end{proof}
\end{lem}

\begin{thm} \ \\
\label{MainInEq}
Let $X\subset\mathbb{R}^n$ be closed and $\bold{f}:\mathbb{R}^n\rightarrow\mathbb{R}^n$ be continuous s.t. $\bold{f}(X)$ contains the origin. If $\bold{f}$ has no singularities then $$\min\limits_{||\bold{\lambda}||=1}\max\limits_{x\in X}\ \bold{\lambda}^T\bold{f}(x)\geq \min\limits_{x\in \partial X}\max\limits_{||\bold{\lambda}||=1}\ \bold{\lambda}^T\bold{f}(x)$$
\begin{proof} \ \\
If the origin lies on the boundary of $\bold{f}(X)$ then clearly $$\min\limits_{||\bold{\lambda}||=1}\max\limits_{x\in X}\ \bold{\lambda}^T\bold{f}(x)\geq 0 = \min\limits_{x\in \partial X}\max\limits_{||\bold{\lambda}||=1}\ \bold{\lambda}^T\bold{f}(x)$$
Thus assume the origin lies in the interior of $\bold{f}(X)$.
Let $\bold{\hat{x}}$ be the point and $\bold{\lambda}_{\bold{\hat{x}}}$ the unit vector at which $$\min\limits_{||\bold{\lambda}||=1}\max\limits_{x\in X}\ \bold{\lambda}^T\bold{f}(x)$$
obtains its optimal. \\

Case 1: If the angle, $\theta$, between $\bold{\hat{x}}$ and $\bold{\lambda}_{\bold{\hat{x}}}$ is 0 then $\max\limits_{||\bold{\lambda}||=1}\bold{\lambda}^T\bold{f}(\bold{\hat{x}})=\bold{\lambda}_{\bold{\hat{x}}}^T\bold{f}(\bold{\hat{x}})$. Furthermore by the previous lemma $\bold{f}(\bold{\hat{x}})\in \partial \bold{f}(X)$ and thus $\bold{\hat{x}}\in \partial X$ since $\bold{f}$ has no singularities by hypothesis. It follows now that $$\min\limits_{x\in \partial X}\max\limits_{||\bold{\lambda}||=1}\ \bold{\lambda}^T\bold{f}(x)\leq \bold{\lambda}_{\bold{\hat{x}}}^T\bold{f}(\bold{\hat{x}})=\min\limits_{||\bold{\lambda}||=1}\max\limits_{x\in X}\ \bold{\lambda}^T\bold{f}(x)$$

Case 2: Assume $\theta \neq 0$. Let $\bold{x^*}$ be a point on the boundary of $X$ s.t. the angle between $\bold{\lambda}_{\bold{\hat{x}}}$ and $\bold{f}(\bold{x^*})$ is 0. Such a point must exist as $\bold{f}(X)$ is compact and by hypothesis $\bold{f}(X)$ contains the origin, has no singularities ($\bold{f}$ maps $\partial X$ to $\partial \bold{f}(X)$) and by assumption the origin lies in the interior of $\bold{f}(X)$. It follows then that $$\min\limits_{x\in \partial X}\max\limits_{||\bold{\lambda}||=1}\ \bold{\lambda}^T\bold{f}(x)\leq \max\limits_{||\bold{\lambda}||=1}\bold{\lambda}^T \bold{f}(\bold{x^*}) =\bold{\lambda}_{\bold{\hat{x}}}^T\bold{f}(\bold{x^*})\leq \max\limits_{x\in X}\bold{\lambda}_{\bold{\hat{x}}}^T\bold{f}(x)=\min\limits_{||\bold{\lambda}||=1}\max\limits_{x\in X}\ \bold{\lambda}^T\bold{f}(x)$$

The theorem now follows.

\end{proof}
\end{thm}
\ \\
If \eqref{eq:RSForm} is invalidated then $\exists \hat{x}\in\partial\Omega$ s.t. $F(\hat{x})=\hat{u}$ for some $\hat{u}\in[u_{min},u_{max}]$ and thus by Theorem \ref{MainInEq} 
$$\min\limits_{||\bold{\lambda}||=1}\max\limits_{x\in \bar{\Omega}}\ \bold{\lambda}^T\left(\bold{F}(x)-\hat{u}\right)\geq 0 = \min\limits_{x\in \partial \Omega}\max\limits_{||\bold{\lambda}||=1}\ \bold{\lambda}^T\left(\bold{F}(x)-\hat{u}\right).$$
Note that under these circumstances $\min\limits_{||\bold{\lambda}||=1}\max\limits_{x\in \bar{\Omega}}\ \bold{\lambda}^T\left(\bold{F}(x)-\hat{u}\right)> 0$ is possible, especially if $F(\bar{\Omega})$ is highly non-convex. These results validate the inner and outer bound approximation models we now present.





 



\section{Numerical studies}


\section{Conclusion}
\end{document}
