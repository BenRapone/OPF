\section{Introduction} \label{sec:intro}
  
\todo{Add detailed intro points from the grant proposal.}

This paper studies quadratic systems of equations with parameters.
In recent years, several algorithms have been proposed for the solution of quadratic systems, along with analyses of conditions under which they work. 
However, in many applications, parameters appearing in the quadratic system of equations are not known perfectly.
In these cases, it is of interest to study robust solvability of quadratic systems, that is, whether the quadratic system of equations has a solution (within a certain set) for all realizations within a certain error bound of a given nominal value of the parameters.
  More concretely, we study a system of $n$ quadratic equations $F(\bold x)=\bold u$ where $F: \mathbb{R}^n \mapsto \mathbb{R}^n,\bold  x,\bold u \in \mathbb{R}^n$ and $F$ is quadratic in $\bold x$.
  We are interested in situations where the parameters $\bold u$ are uncertain and we are still interested in guaranteeing that there is a solution to $F(\bold x) = \bold u$ for $\bold x$ within limits on $\bold x$ and $\bold u$.
  Questions of this type arise in a variety of applications from analysis of stochastic processes to infrastructure networks like the power grid and the gas grid.

\subsection{Our Contributions}
\todo{Informally define robustness margin first---as a new way to characterize robust feasibility.}
We develop approaches based on topological degree theory to estimate bounds on the robustness margin of such systems.
  Our methods use tools from convex analysis and optimization theory to cast the problems of checking the conditions for robust feasibility as a nonlinear optimization problem.
  We then develop \emph{inner bound} and \emph{outer bound} formulations for this optimization problem, which could be solved efficiently to derive lower and upper bounds, respectively, for the margin of robust feasibility.
  We evaluate our approach numerically on standard instances taken from the MATPOWER database of AC power flow equations that describe the steady state of the power grid.
  The results demonstrate that our approach can produce tight lower and upper bounds on the radius of robust feasibility for such instances.

\subsection{Related Work}

\todo{Add paragraph about related paper "Solvability Regions of Affinely Parameterized Quadratic Equations''.
Also check for any new/recent papers/preprints presenting related approaches.}
  
Robust feasibility and optimization have been well-studied by both the optimization and topology communities. 
What is lacking is an approach that can gaurantee and quantify robust feasibility on large scale systems in an efficient manner. 
In this article we address this deficiency by developing theory that utilizes results from topological degree theory and convex optimization. 
We provide a theoretical foundation for determining robust feasibility of systems of quadratic equations and computational methods for producing lower and upper bounds on the maximum error bound for which one can guarantee robust solvability (the radius of robust solvability). 
To highlight the efficacy of our approach we derive models, which we test numerically on several quadratic systems constructed from the AC power flow equations that describe the steady state of the power grid with added uncertainty. 
The results show that our approach can be applied to large scale systems to produce tight lower and upper bounds on the radius of robust solvability, which we shall define as the robustness margin of the system.

In optimization, the focus has been on robust \emph{convex} optimization where uncertainty sets are specified for the parameters of a convex optimization problem (typically an LP or conic program) \cite{ben2009robust}.
Robust \emph{nonconvex} optimization has received only limited attention (a notable exception is the work of Bertsimas et al.~\cite{BeNoTe2010}).
These approaches do not provide rigorous guarantees for robust feasibility with nonconvex constraints.

In algebraic topology, there have been a number of studies on these problems based on several approaches, including ones based on robustness of level sets and persistent homology \cite{BeEdMoPa2010,EdMoPa2011}, well groups and diagrams \cite{ChSkPa2012,FrKr2016well,FrKr2016pers}, topological degree and robust satisfiability \cite{FrKr2015,FrKrWa2016},  and on Borsuk's theorem and interval arithmetic \cite{FrRa2015,FrHoLa2007,FrLa2005}.
While the theory developed by these approaches is fairly complete, the associated algorithms typically rely on explicit simplicial or cellular decompositions of the problem space.
But the size of such decompositions typically grows exponentially in the problem dimension, and hence these algorithms are typically impractical for large-scale applications.

Looking specifically at applications such as the power systems, there has been significant interest in solving the non-robust version of the OPF problem to global optimality.
The driver has been the development of strong convex relaxations of the nonconvex optimization problems combined with ideas from global optimization such as spatial branch-and-cut, bound tightening, etc.~\cite{BiMu2016,coffrin2015strengthening}.
Uncertainty has been handled in a chance-constrained framework \cite{BiChHa2014,zhang2011chance}.
However, this approach has typically been applied only to linear approximations or convex relaxations of the AC power flow equations, and does not guarantee feasibility with respect to the true nonlinear power flow equations \cite{BiChHa2014,kocuk2016strong,RoVrOlAn2015,TsBiTa2016}.
