\section{Computing Lower Bounds on the Robustness Margin} \label{sec:inbdform}

In this section we will derive procedures for computing a lower bound on the robustness margin. 
We start with an exact formulation, which turns out to be hard to implement efficiently in practice.
Hence we relax the procedures until they become computationally tractable. 
We end the section by providing three different implementations of our final derived, relaxed, computationally tractable procedure specified in Equation (\ref{eq:OPTfeasrelax}).
Each of these three practical implementations brings a unique set of attributes, which makes none of them the clearly preferred candidate.

\smallskip
\begin{thm}
  Let $\Omega=\{\vx| A\vx \leq \vb\}$, $\Omega_{u}=\{\vu| u^{\min}_i\leq u_i \leq u^{\max}_i \ \forall i \}$, and $F(\vx)=Q(\vx)+L\vx$ as described in Equations (\ref{eq:Quad}), (\ref{eq:xLimits}), and (\ref{eq:uLimits}). 
  Let
  \[
  z = \min\limits_{\vx\in \partial \Omega, \vu\in \Omega_{u} }\max\limits_{\norm{\vlambda}=1}\ \vlambda^T\left(F(\vx)-\vu\right).
  \]
  If there is an $r>0$ such that $r \leq e_i \ \forall i$ with $\ e_i>0$, where $e_i$ denotes the error bounds associated with $ u^{\min}_i$ and $ u^{\max}_i$, and if $z > 0$ then the system is robust feasible and has a robustness margin of at least $r$.

  \medskip
  \begin{proof} 
    If \cref{eq:RSForm} is invalidated then there exists $\hat{\vx} \in \partial\Omega$ such that $F(\hat{\vx})=\hat{\vu}$ for some $\hat{\vu}\in\Omega_{u}$ and thus
    \[
    z = \min\limits_{\vx\in \partial \Omega}\max\limits_{\norm{\vlambda}=1}\ \vlambda^T\left(F(\vx)-\vu\right) ~\leq~ \max\limits_{\norm{\vlambda}=1, \vx=\hat{\vx}}\ \vlambda^T\left(F(\vx)-\hat{\vu}\right) ~=~ 0.
    \]
    Hence if $z>0$, \cref{eq:RSForm} is validated, and the system is robust feasible.
    It follows then by definition that the system has a robustness margin of at least $r$.
  \end{proof}
\end{thm}

  \medskip
\begin{thm} \label{thm:RobFeas}
  Let $\partial\Omega_i=\{\vx| (A\vx)_i = b_i, A\vx\leq \vb\}$, $\Omega_{u}=\{\vu| u^{\min}_i\leq u_i \leq u^{\max}_i \ \forall i \}$, and $F(\vx)=Q(\vx)+L(\vx)$ as described in \cref{eq:Quad,eq:xLimits,eq:uLimits}.
  Define
  \begin{align}
    \rz_i =  \min_{\vx\in\partial\Omega_i, \vu \in \Omega_u} \norm{F(\vx)-\vu}. \label{eq:OPTfeas}
  \end{align}
  The system is robust feasible if and only if $\rz_i>0$ for each $i = 1, \ldots, m$, where $m$ is the number of rows of $A$.
%If $\rz_i>0$ for each $i = 1, \ldots, m$ (where $m$ is the number of rows of $A$), then the system is robust feasible.

  \begin{proof} \ \\
    $\boxed{\Rightarrow}$ \\ 
    If the system is not robust feasible then there exists $\hat{\vu}'\in \Omega_u$ such that $F(\vx)=\hat{\vu}'$ has no solution. 
    Since $F(\vx)=\vu^*$ has a solution and $F$ is continuous over a compact domain, there must exist an $\hat{\vx} \in \Omega = \{\vx| A\vx \leq \vb\}$ such that $F(\hat{\vx})=\hat{\vu}$ for some $\hat{\vu}\in \Omega_u\cap \partial F(X)$. 
    Since $\partial F(X) = F(\partial X)$ we have that $\hat{\vx}\in\partial\Omega$, which implies that there is an $i$ such that $A\hat{\vx}_i=b_i$, and thus $0 \leq \rz_i \leq \norm{F(\hat{\vx})-\hat{\vu}}=0$.\\
    %
    $\boxed{\Leftarrow}$ \\ 
    If there exists a $i$ such that $z_i=0$, then it follows that there must be the $\hat{x}\in \partial \Omega$ which contains $\hat{x_i}$ as an element such that $F(\hat{x})=\hat{u}$ for some $\hat{u}\in \Omega_u$. Thus there is no interior point $x$ to make the equation $F(x)=\hat{u}$ holds. Then based on \cref{RobustDef}, the system is not robust feasible. 
    %If there exists an $i$ such that $\rz_i = 0$, then there must exist an $\hat{\vx} \in \partial\Omega $ and $\hat{\vu}\in \Omega_u$ such that $F(\hat{\vx})=\hat{\vu}$. 
    %Since $\partial F(X) = F(\partial X)$, there cannot exist an $\hat{\vx}' \in \Int\Omega$ such that $F(\hat{\vx}')=\hat{\vu}$, and hence by definition the system is not robust feasible.
    %\cref{eq:RSForm} is invalidated, which implies $\exists \hat{\vx}\in\partial\Omega$, $\Omega=\{\vx| A\vx< \vb\}$, such that $F(\hat{\vx})=\hat{\vu}$ for some $\hat{\vu}\in \Omega_u$. 
    %Since $\hat{\vx}\in\partial\Omega$ this implies that $\exists i$ such that $A\hat{\vx}_i=b_i$ and thus $0\leq \rz_i \leq ||F(\hat{\vx})-\hat{\vu}||=0$.
    %Thus if $\rz_i>0$ for each $i = 1, \ldots, m$, \cref{eq:RSForm} is validated, and the system is robust feasible.
  \end{proof}
\end{thm}

%Note that under these circumstances $\min\limits_{\norm{\vlambda}=1}\max\limits_{x\in \bar{\Omega}}\ \vlambda^T\left(F(x)-\hat{u}\right)> 0$ is possible, especially if $F(\bar{\Omega})$ is highly non-convex. However, should $\exists u^{\prime}$ such that $\min\limits_{\norm{\vlambda}=1}\max\limits_{x\in \bar{\Omega}}\ \vlambda^T\left(F(x)-u^{\prime}\right)=0$ then by \ref{MainInEqLem} we know that there $\exists \hat{x}\in\partial\Omega$ such that $F(\hat{x})=u^{\prime}$. Hence the following formulation of an outer bound approximation on $u_i^{\min}$ and $u_i^{\max}$:


The optimization problems presented in \cref{thm:RobFeas} are nonlinear and nonconvex.
Hence it may be difficult to solve them in general.
In fact, since $F(\vx)$ is quadratic in $\vx$, they are quadratically constrained quadratic programs (QCQPs), which are NP-hard in general \cite{PaBo2017}.
At the same time, we can use well known \emph{semidefinite programming relaxations for QCQPs} to obtain lower bounds on the optimal values \cite{VaBo1996}.
Since $Q$ is quadratic, it can also be written as a linear function of $\vx\vx^T$.
More concretely, we can write
$$Q(\vx)_i=\vx^TQ_i\vx=\Tr(Q_i\vx\vx^T)$$
where each $Q_i$ is as defined in \cref{eq:Quad}. 
Since $\vx$ should satisfy $A\vx\leq \vb$, we get
\[
  (\vb-A\vx)(\vb-A\vx)^T \geq 0~~\Rightarrow~~ \vb\vb^T-A\vx\vb^T-\vb(A\vx)^T+A(\vx\vx^T)A^T\geq 0.
\]
If we allow a symmetric positive semidefinite matrix $X$ to take the place of $\vx\vx^T$ and drop the rank constraint ($\rank(X)=1$)) we can construct the following relaxation for the optimization problem presented in \cref{thm:RobFeas}:
% 
\begin{equation}\label{eq:OPTfeasrelax}
  \begin{array}{rl}
    \hat{\rz}_{i} = & \min\limits_{\vx}  b_i-(A\vx)_i  \\
    \vspace*{-0.05in} \\
%\begin{align*}
    \text{subject to } \ & \Tr\left(Q_iX\right)+ L_i\vx \geq \vu_i^{\min} \ \ \forall i\\
    & \Tr\left(Q_iX\right)+ L_i\vx \leq \vu_i^{\max} \ \ \forall i\\
    &A\vx\leq \vb \\
    &\vb\vb^T-A\vx\vb^T-\vb(A\vx)^T+AXA^T\geq O \\
    &X \text{ is symmetric and positive semidefinite.}
  \end{array}
\end{equation}
%\end{align*}
%
Here $L_i$ denotes the $i^{\rm th}$ rows of $L$, and $O$ denotes the $2n \times 2n$ matrix of zeros with the constraints understood to be component-wise inequalities.
Note that if we impose $\rank(X)=1$, we do get an exact formulation.
At the same time, it may be difficult to impose this constraint.
But the system (in \cref{eq:OPTfeasrelax}) without the rank constraint is a convex optimization problem (a semidefinite program, in fact) and can be solved efficiently. 
Since this is a relaxation of the procedure described in \cref{thm:RobFeas}, if $\hat{\rz}_i>0$ for each $i$, the condition in \cref{thm:RobFeas} (i.e., $\rz_i>0 \ \forall i$) is satisfied. 
We can further relax the formulation in \cref{eq:OPTfeasrelax} by dropping the condition that $X$ be positive semidefinite, which transforms the problem from a semidefinite program to a linear program. 
We now present three formulations for this new, relaxed program, and provide numerical results for each of them in \cref{ssec:compres}.
We also describe the advantages and drawbacks of each formulation.

\bigskip
\bigskip
\textbf{LP Feasibility Procedure} 
\begin{equation} \label{eq:OPTfeasrelaxLP1}
\begin{array}{rl}
 &\text{Find an }\vx \\
 \text{subject to } \ &(A\vx)_i= b_i \\
 &\Tr\left(Q_iX\right)+ L_i\vx \geq \vu_i^{\min}  \ \ \forall i\\
 & \Tr\left(Q_iX\right)+ L_i\vx \leq \vu_i^{\max}  \ \ \forall i\\
 	&A\vx\leq \vb \\
 	&\vb\vb^T-A\vx\vb^T-\vb(A\vx)^T+AXA^T\geq O \\
 	&X \text{ is symmetric.}
\end{array}
\end{equation}


This procedure has the advantage of being a linear program and hence can be solved efficiently.
But as noted by the objective $\rz_i$, one must iterate over each dimension of $A\vx$ checking feasibility of the procedure. 
Of course, should the procedure prove feasible then we have found a solution on the boundary and thus invalidating the Statement in (\cref{eq:RSForm}). 
If the procedure proves infeasible then we are free to push the robustness margin higher and test again. 
An alternative approach is to consider all of the dimensions of $A\vx$ simultaneously by introducing extra binary variables and creating a MIP as follows. 

\bigskip
\textbf{MIP Procedure}\
\begin{equation}\label{eq:OPTfeasrelaxLP2}
\begin{array}{rl}
\max &  z  \\
 \text{subject to } \ &\Tr\left(Q_iX\right)+ L_i\vx \geq \vu_i^{\min}  \ \ \forall i\\
 & \Tr\left(Q_iX\right)+ L_i\vx \leq \vu_i^{\max}  \ \ \forall i\\
 	&A\vx\leq \vb \\
 	&\vb\vb^T-A\vx\vb^T-\vb(A\vx)^T+AXA^T\geq O \\
 	&X \text{ is symmetric} \\
 	& z\leq (A\vx)_i-b_i+R(1-d_i) \ \forall i \ \text{ for some large enough $R$} \\
 	& \sum\limits_i d_i=1 \\
 	& d_i\in\{0,1\} \ \forall i .
\end{array}
\end{equation}


The MIP and LP Feasibility procedures are similar and should theoretically give the same results. 
However, as we will show in our numerical studies, the LP Feasibility procedure could outperform the MIP procedure by producing higher robustness margins and running faster in higher dimensions. 
In both procedures, the process ends after a boundary solution is found. 
This solution may not be a boundary solution to the actual system, but may be an artifact of the relaxations used to create the procedures. 
One way of tackling this issue is by updating the constraints using an iterative process as outlined in the following procedure.

\bigskip
\textbf{LP Bound Tightening Procedure}
\begin{equation}\label{eq:OPTfeasrelaxLP3}
\begin{array}{rl}
\max &  z_i = (A\vx)_i  \\
 \text{subject to } \ &\Tr\left(Q_iX\right)+ L_i\vx \geq \vu_i^{\min}  \ \ \forall i\\
 & \Tr\left(Q_iX\right)+ L_i\vx \leq \vu_i^{\max}  \ \ \forall i\\
 	&A\vx\leq \vb \\
 	&\vb\vb^T-A\vx\vb^T-\vb(A\vx)^T+AXA^T\geq O \\
 	&X \text{ is symmetric.}
\end{array}
\end{equation}

What distinguishes the LP Bound Tightening procedure from the other two procedures is the ability to use it iteratively by updating the constraints of \cref{eq:OPTfeasrelaxLP3}, replacing $\vb$ with $\vz$, the vector of $z_i$'s found after running the procedure over all dimensions of $A\vx$. 
Since clearly $\vz \leq \vb$, we have that the polytope $\{\vx ~|~ A\vx\leq \vz\} \, \subseteq \, \{\vx ~|~ A\vx\leq \vb\}$. 
Thus it follows that any system deemed robust feasible using \cref{eq:OPTfeasrelaxLP1} or \cref{eq:OPTfeasrelaxLP2} will certainly be found robust feasible using \cref{eq:OPTfeasrelaxLP3}, but a system found robust feasible using \cref{eq:OPTfeasrelaxLP3} may not be found robust feasible using \cref{eq:OPTfeasrelaxLP1} or \cref{eq:OPTfeasrelaxLP2}. 
The drawback of the bound tightening procedure, as we shall see in the Computational Results (\cref{ssec:compres}), is choice of parameters to be manually set in order to tell the procedure when to stop. 

